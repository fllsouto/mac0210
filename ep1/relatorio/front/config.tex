%!TEX encoding = UTF-8 Unicode

%packages
\usepackage[T1]{fontenc}
\usepackage[utf8]{inputenc}      % Encoding
\usepackage[portuguese]{babel}   % Correção
%\usepackage{caption}             % Legendas
\usepackage{enumerate}

% Matemática
\usepackage{amsmath}             % Matemática
\usepackage{amsthm, amssymb}     % Matemática
\newtheorem*{def*}{Definição}
\newtheorem*{invariant}{Invariante}

% Gráficoshttps://preview.overleaf.com/public/cjyckkjhrcbg/images/b90c5a3de444d37d887a1e2dff4bd557468ec108.jpeg
\usepackage[usenames,dvipsnames]{color}  % Cores
\usepackage[pdftex]{graphicx}   % usamos arquivos pdf/png como figura
\usepackage[usenames,svgnames,dvipsnames]{xcolor}

% Desenhos
\usepackage{tikz}
\usepgfmodule{decorations}
\usetikzlibrary{patterns}
\usetikzlibrary{decorations.shapes}
\usetikzlibrary{shapes.geometric}
\usetikzlibrary{decorations.text}
\usetikzlibrary{positioning} % Adjust grid size

% Código-fonte
\usepackage[noend]{algpseudocode}
\usepackage{algorithm}

% Configurações da página
\usepackage{float}
\usepackage{setspace}           % espaçamento flexível
\usepackage{indentfirst}        % Identa primeiro parágrafo
\usepackage{makeidx}
\usepackage[nottoc]{tocbibind}  % acrescentamos a  bibliografia/indice/
                                % conteudo no Table of bContents

% Fontes
\usepackage{courier}
\usepackage{type1cm}            % fontes realmente escaláveis
\usepackage{titletoc}
\usepackage{pdflscape}          % Páginas em paisagem
\usepackage{pdfpages}

% Fontes e margens
\usepackage[fixlanguage]{babelbib}
\usepackage[font=small,format=plain,labelfont=bf,up,textfont=it,up]{caption}
\usepackage[a4paper,top=2.54cm,bottom=2.0cm,left=2.0cm,right=2.54cm]{geometry}

\usepackage{subcaption}
\usepackage[toc,page]{appendix}

% Referências e citações
\usepackage[
    pdftex,
    breaklinks,
    plainpages=false,
    pdfpagelabels,
    pagebackref,
    colorlinks=true,
    citecolor=DarkGreen,
    linkcolor=DarkBlue,
    urlcolor=DarkRed,
    filecolor=green,
    bookmarksopen=true
]{hyperref}
\usepackage[all]{hypcap} % Soluciona o problema com o hyperref e capitulos
\usepackage[angle,sort]{natbib}

%\bibpunct{(}{)}{;}{a}{\hspace{-0.7ex},}{,}  % Estilo de citação
% \bibpunct{(}{)}{;}{a}{,}{,}

%%%%%%%%%%%%%%%%%%%%%%%%%%%%%%%%%%%%%%%%%%%%%%%%%%%%%%%%%%%%%%%%%%%%%%%%%%%%%%%%%%


%%%%%%%%%%%%%%%%%%%%%%%%%%%%%%%%%%%%%%%%%%%%%%%%%%%%%%%%%%%%%%%%%%%%%%%%%%%%%%%%%%
% Cabeçalhos similares ao TAOCP de Donald E. Knuth
\usepackage{fancyhdr}
\pagestyle{fancy}
\fancyhf{}
\renewcommand{\sectionmark}[1]{\markright{\MakeUppercase{#1}}{}}
\renewcommand{\headrulewidth}{0pt}

% Redefine \emph
\makeatletter
\DeclareRobustCommand{\em}{%
  \@nomath\em \if b\expandafter\@car\f@series\@nil
  \normalfont \else \bfseries \fi}
\makeatother

\usepackage{afterpage}

\newcommand\blankpage{%
    \null
    \newpage}

%%%%%%%%%%%%%%%%%%%%%%%%%%%%%%%%%%%%%%%%%%%%%%%%%%%%%%%%%%%%%%%%%%%%%%%%%%%%%%%%%%
\frenchspacing              % arruma o espaço: i.e. e e.g.
\urlstyle{same}             % URL com o mesmo estilo do texto
                            % e não mono-spaced
\makeindex                  % para o índice remissivo
\raggedbottom               % para não permitir espaços extra no texto
\fontsize{60}{62}\usefont{OT1}{cmr}{m}{n}{\selectfont}
\cleardoublepage
\normalsize

%%%%%%%%%%%%%%%%%%%%%%%%%%%%%%%%%%%%%%%%%%%%%%%%%%%%%%%%%%%%%%%%%%%%%%%%%%%%%%%%%%
% INFORMAÇÕES
\pdfinfo{
  /Title    (Título da Monografia)
  /Author   (Autor)
  /Creator  (Autor)
  /Producer (Autor)
  /Subject  (Trabalho de Conclusão de Curso)
  /Keywords (TCC, BCC, Tema)
}

\usepackage{eso-pic}
\newcommand\AlCentroPagina[1]{%
  \AddToShipoutPicture*{\AtPageCenter{%
  \makebox(0,0){\includegraphics%
  [width=0.6\paperwidth]{#1}}}}}

\newcommand{\monoAutor}{Fellipe Souto Sampaio}
\newcommand{\monoOrientadora}{Prof Dra Ana Cristina Viera de Melo}
\newcommand{\monoUniversidade}{Universidade de São Paulo, Brasil}
\newcommand{\monoObj}{Dissertação para obteção do título de Bacharel em Ciência da Computação}
\newcommand{\monoTitulo}{Um estudo do Modelo de Atores aplicado à concorrência de recursos}
\newcommand{\monoData}{01 de Dezembro de 2016}
\newcommand{\monoAno}{2016}

\usepackage{fancyhdr}
\pagestyle{empty}

\newenvironment{abstract}%
{\clearpage\null \begin{center}%
\bfseries\abstractname \end{center}}%
{\vfill\null}

\usepackage{epigraph}

\setlength\epigraphwidth{.8\textwidth}

\newenvironment{dedication}
  {%\clearpage           % we want a new page          %% I commented this
   \thispagestyle{empty}% no header and footer
   \vspace*{\stretch{1}}% some space at the top
   \raggedleft          % flush to the right margin
  }
  {\par % end the paragraph
   \vspace{\stretch{3}} % space at bottom is three times that at the top
   \clearpage           % finish off the page
  }