%%%%%%%%%%%%%%%%%%%%%%%%%%%%%%%%%%%%%%%%%%%%%%%%%%%%%%%%%%%%%%%%%%%%%%%%%%%%%%%%
%%                                 Parte 2                                    %%
%%%%%%%%%%%%%%%%%%%%%%%%%%%%%%%%%%%%%%%%%%%%%%%%%%%%%%%%%%%%%%%%%%%%%%%%%%%%%%%%

%%%%%%%%%%%%%%%%%%%%%%%%%%%%%%%%%%%%%%%%%%%%%%%%%%%%%%%%%%%%%%%%%%%%%%%%%%%%%%%%
\chapter{Método de Newton}\label{partes:newtonMethod}

  A implementação do nosso Método de Newton se baseou na disponível em \cite{numericalMethods}. Para calcular a sequência de valores $X_k$ aplicamos o seguinte pseudo-algoritmo:

  \begin{algorithm}
    \caption{Método de Newton}\label{euclid}
    \begin{algorithmic}[1]
    \Procedure{Newton}{}
      \State {$ X_K \gets \infty + i $}
      \State {$ X_{k+1} \gets X_K $}
      \State {$ mx\_it \gets 15 $}
      \State {$ \delta \gets 1.0 \times 10^{-8} $ \\}
      \While{$abs(X_{k + 1} - X_{k} > \delta) || (it < mx\_it)$}
        \State {$ it++ $}
        \State {$ X_K = X_{k + 1} $}
        \If {$ f^{'}(X_k) == 0 $}
          \State {O método não está definido para esse ponto, então consideramos que ele falhou}
          \State{\Return{$\infty + \infty \times i$}}
        \Else
          \State {$X_{k + 1} \gets X_K - \frac{f(X_k)}{f^{'}(X_k)}$}\\
          \If {$ it > mx\_it $}
            \State {O método não está convergindo, então consideramos que ele falhou}
            \State{\Return{$\infty + \infty \times i$}}
          \EndIf
        \EndIf
      \EndWhile
      \State {A raíz encontrada é $X_K = X + Y \times i$}
      \State{\Return{$X + Y \times i$}}
    \EndProcedure
    \end{algorithmic}
  \end{algorithm}

  Escolhemos um número fixo de iterações igual a 15, em testes práticos demonstrou-se que a função convergia em no máximo 12 iterações. Repetimos esse algoritmo para todos os pontos da nossa partição escolhida, e para cada resposta fazemos um mapeamento da raíz para um inteiro. \\

  Um polinômio de grau N com raízes distintas mapeia N+1 cores diferentes, a cor adicional, neste caso preto, serve para identificar os pontos em que o método falhou.

  Realizamos testes com diferentes funções, variando a quantidade de pixeis na imagem. Os gráficos podem ser encontrados no Apêndice (\ref{chap:appendix}) desse relatório.

%%%%%%%%%%%%%%%%%%%%%%%%%%%%%%%%%%%%%%%%%%%%%%%%%%%%%%%%%%%%%%%%%%%%%%%%%%%%%%%%
