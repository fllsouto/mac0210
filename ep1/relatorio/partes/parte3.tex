%%%%%%%%%%%%%%%%%%%%%%%%%%%%%%%%%%%%%%%%%%%%%%%%%%%%%%%%%%%%%%%%%%%%%%%%%%%%%%%%
%%                                 Parte 3                                    %%
%%%%%%%%%%%%%%%%%%%%%%%%%%%%%%%%%%%%%%%%%%%%%%%%%%%%%%%%%%%%%%%%%%%%%%%%%%%%%%%%

%%%%%%%%%%%%%%%%%%%%%%%%%%%%%%%%%%%%%%%%%%%%%%%%%%%%%%%%%%%%%%%%%%%%%%%%%%%%%%%%
\chapter{Encontrando Todas as Raízes de Funções}\label{partes:allRootsFunction}

  No enunciado da terceira questão é mencionado sobre uma função f(x) de classe $C^2$ definida em um intervalo [a, b]. Foi, então, suposto por nós que a e b são números reais, e por ser de classe $C^2$, a função f(x) é garantidamente contínua.\\

  Para usar o programa, é necessário fazer as devidas modificações:\\
  \begin{enumerate}
    \item Mudar o valor da variável \textit{choice} para um dos valores: 1, 2 ou 3.
    \begin{enumerate}
      \item Para choice = 1, o programa considerará a função $f(x) = 2cosh(\frac{x}{4}) - x$
      \item Para choice = 2, o programa considerará a função $f(x) = \frac{sin(x)}{x}$ se $x \neq 0$ ou $f(x) = 1$ se $x = 0$
      \item Para choice = 3, o programa considerará um polinomio, que deve ser escrito de acordo com a sintaxe do Octave, na função \textit{function3}, no código do programa. Convém que a função que representa o polinomia seja, no mínimo, de classe $C^2$, conforme exigência do enunciado (isso fica a cargo do usuário e não do programa).
    \end{enumerate}
    \item Mudar os valores das variáveis \textit{lower\_edge} e \textit{higher\_edge} (estas fazem os papeis de a e b do enunciado, respectivamente).
    \item Mudar o valor da variável n\_inter para o número de partições (sub-intervalos) em que o intervalo [a, b] deve ser dividido.
    \item Mudar a tolerância \textit{tol} do erro para aceitar o valor calculado pelo método de Newton como significativamente próximo ao valor real da raíz.
  \end{enumerate}

  Finalmente, atribuídos os valores iniciais, o programa vai realizar a tarefa que foi designada a nós no enunciado da seguinte maneira:\\
  \begin{itemize}
  \item{Será invocado o método \textit{intervals}, que é onde tudo acontecerá. Deve-se escolher o tamanho dos passos (= steps, caso não goste do que está escrito no código)}
  \item{A variável \textit{range} conterá todos os pontos x de f(x) que deverão ser testados para encontrar a raíz}
  \item{A variável \textit{sub\_interval} serve para monitorar qual é o número do intervalo em que o ponto x que está sendo testado está}
  \item{A variável \textit{sub\_inter\_length} guarda o tamanho das partições, enquanto as variáveis \textit{sub\_lower\_edge} e \textit{sub\_higher\_edge}
  guardam os limites da participação atual (em que o programa se encontra em tempo de execução)}
  \end{itemize}

  Note que para um intervalo [a, e] dividido em 4 partes, as partições de [a, e] são [a, b]; ]b, c]; ]c, d], ]d, e] respectivamente, além disso apenas a primeira partição possui o inteiro limitante inferior.\\
  
  Em seguida, o primeiro ponto do nosso intervalo é avaliado em f(x) separadamente para obtermos o sinal inicial de f(x). Se dermos sorte e o primeiro valor já for zero, significa que primeiro ponto é uma raiz. Neste caso, haverá uma impressão do intervalo, o número do intervalo e o valor da raíz no terminal. Note que valor 1 representa sinal não negativo e o valor 2 representa sinal negativo.\\

  Obtido o sinal, o programa entra em um loop que vai do segundo ao penúltimo valor de x. Calculamos o novo valor $f(x_1)$, considerando que $x_1$ é o ponto seguinte da sequência após x. Com isso temos os possíveis resultados:\\

  \begin{itemize}
  \item{Se $f(x_1) = 0$ então uma raíz foi encontrada em $x_1$, e nesse caso o sinal é considerado positivo, as informações são impressar e o laço já recomeça a próxima iteração, sem rodar o método de Newton}
  \item{Se houver uma mudança de sinal de f(x) para $f(x_1)$ significa que existe uma raíz entre esses dois pontos, e nesse caso aplicamos o método de Newton para buscar uma aproximação do valor real da raiz considerando a tolerância tol}
  \end{itemize}

  Se a cada iteração o decrescimento não for significativo (isto é, nosso ponto de partida x não é tão bom e, portanto, não estamos conseguindo aproximações expressivas do valor real da raíz), o método de Newton será interrompido e um novo valor do x atual será buscado pelo método da biseção (que fará 3 iterações apenas, conforme o enunciado pede).\\

  Encontrado o novo valor de x pelo método da biseção, usamos este valor como entrada do método de Newton e assim obteremos, eventualmente, a nova raíz até então desconhecida. Ao encontrar a raíz, assim como nas vezes anteriores, será impresso no terminal as suas informações.\\

  Isso é repetido até que chegamos, finalmente, ao final da iteração do penúltimo valor de x a ser testado, finalizando assim o loop principal do programa. Após isso é checado separadamente o último valor de x, considerando o penúltimo valor de x.\\

  O código pertinente a esta parte está bastante comentado, então entendendo esta explicação e acompanhando o código deve ser mais que o suficiente para entender a implementação sem maiores dificuldades.\\

  Realizamos a execução do programa para as três possíveis choises do programa, os resultados dos testes podem ser encontrado no Apêncide (\ref{section:allRoots}) desse relatório. O programa foi implementado no arquivo \textbf{n\_roots\_function.m}


%%%%%%%%%%%%%%%%%%%%%%%%%%%%%%%%%%%%%%%%%%%%%%%%%%%%%%%%%%%%%%%%%%%%%%%%%%%%%%
