%%%%%%%%%%%%%%%%%%%%%%%%%%%%%%%%%%%%%%%%%%%%%%%%%%%%%%%%%%%%%%%%%%%%%%%%%%%%%%%%
%%                                 Parte 1                                    %%
%%%%%%%%%%%%%%%%%%%%%%%%%%%%%%%%%%%%%%%%%%%%%%%%%%%%%%%%%%%%%%%%%%%%%%%%%%%%%%%%

%%%%%%%%%%%%%%%%%%%%%%%%%%%%%%%%%%%%%%%%%%%%%%%%%%%%%%%%%%%%%%%%%%%%%%%%%%%%%%%%
\chapter{Aritmética de Ponto Flutuante}\label{partes:floatPointArit}


  %%%%%%%%%%%%%%%%%%%%%%%%%%%%%%%%%%%%%%%%%%%%%%%%%%%%%%%%%%%%%%%%%%%%%%%%%%%%%%
  \section{Questão 1}

  Seja: \\

  $X = \pm S \times 2^{E}$, onde:

  \begin{itemize}
  \item $S = (0.b_2 b_3 b_4 \dots b_24)$
  \item $\frac{1}{2} < S < 1$
  \item $-128 < E < 127$ 
  \end{itemize}
   
  \label{sec:question1}
  \subsection{A}

    Para que $X$ seja o maior número de ponto flutuante do sistema precisamos que três condições sejam verdade:

    \begin{enumerate}
    \item{S tem que ter a maior mantissa possível}
    \item{E tem que ter o maior valor possível}
    \item{$X$ tem que ser positivo}
    \end{enumerate}

    Então $X = +(0.111 \dots 1) \times 2^{126}$ é o maior número positivo de ponto flutuante do sistema

  \subsection{B}

    Para que $X$ seja o menor número positivo de ponto flutuante do sistema precisamos que três condições sejam verdade:

    \begin{enumerate}
    \item{S tem que ter a menor mantissa possível, de forma que $\frac{1}{2} < S < 1$}
    \item{E tem que ter o menor valor possível}
    \item{$X$ tem que ser positivo}
    \end{enumerate}

      Então $X = +(0.100 \dots 1) \times 2^{-127}$ é o menor número positivo de ponto flutuante do sistema

  \subsection{C}

    Para resolver essa questão podemos tentar escrever os primeiros números inteiros no dado sistema

    \begin{itemize}
    \item{$X_1 = +(0.100 \dots 0) \times 2^{1} = 1.000 \dots 0 = 1$}
    \item{$X_2 = +(0.100 \dots 0) \times 2^{2} = 10.000 \dots 0 = 2$}
    \item{$X_3 = +(0.110 \dots 0) \times 2^{2} = 11.000 \dots 0 = 3$}
    \item{$X_4 = +(0.100 \dots 0) \times 2^{3} = 100.000 \dots 0 = 4$}
    \end{itemize}

    Entretanto verificando a definição de S vemos que $\frac{1}{2} < S < 1$, portanto para ser um número válido no sistema $S \neq (0.100 \dots 0)$. Concluimos que o menor inteiro que não é exatamente representável no sistema é o número 1.

  %%%%%%%%%%%%%%%%%%%%%%%%%%%%%%%%%%%%%%%%%%%%%%%%%%%%%%%%%%%%%%%%%%%%%%%%%%%%%%
  \section{Questão 2}
  \label{sec:question2}

    Na precisão single temos que $ p = 23$. \\
    \subsection{A}

      Seja: $ X = (\frac{1}{10})_{10} = (0.000 1100 1100 \dots)_{2} \times 2^{0} = (1.100 1100 1100 \dots)_{2} \times 2^{-4}$

      \subsubsection*{Round down}
        Para $X_{-} $ truncarmos a dízima no $b_{p-1}$, ou seja no vigésimo segundo dígito da mantissa. Temos então: \\

        $round(X) = X_{-} = (1.100 1100 1100 1100 1100 110)_{2} \times 2^{-4}$

      \subsubsection*{Round up}
        Para $X_{+} $ adicionamos 1 no vigésimo segundo dígito da mantissa. Temos então: \\

        $round(X) = X_{+} = (1.100 1100 1100 1100 1100 111)_{2} \times 2^{-4}$

      \subsubsection*{Round towards zero}
        Como $X = \frac{1}{10} > 0$ então $round(X) = X_{-}$

      \subsubsection*{Round to nearest}
        Como $X_{-} < X < X_{+}$ e $X_{-} - X < X_{+} - X $ verificamos que X está mais próximo de $X_{-}$, portanto $round(X) = X_{-}$\\

    \subsection{B}
      Seja $ X = (1 + 2^{-25}) = (1.000 000 000 000 000 000 000 000 1)_2 \times 2^{0}$

      \subsubsection*{Round down}
        Para $X_{-} $ truncarmos a dízima no $b_{p-1}$, ou seja no vigésimo segundo dígito da mantissa. Temos então: \\

        $round(X) = X_{-} = (1.000 000 000 000 000 000 000 0)_{2} \times 2^{0}$
      \subsubsection*{Round up}
        Para $X_{+} $ adicionamos 1 no vigésimo segundo dígito da mantissa. Temos então: \\

        $round(X) = X_{+} = (1.000 000 000 000 000 000 000 1)_{2} \times 2^{0}$

      \subsubsection*{Round towards zero}
        Como $X = (1 + 2^{25}) > 0$ então $round(X) = X_{-}$

      \subsubsection*{Round to nearest}
        Como $X_{-} < X < X_{+}$ e $X_{-} - X < X_{+} - X $ verificamos que X está mais próximo de $X_{-}$, portanto $round(X) = X_{-}$

    \subsection{C}
      Seja $ X = (2^{130}) = (1.000 000 000 000 000 000 000 000 0)_2 \times 2^{130}$

      \subsubsection*{Round down}
        No formato single temos que $N_{max} \approx 2^{128}$, como nosso E é maior que o limite temos então: \\

        $round(X) = N_{max}$
      \subsubsection*{Round up}
        Como $X_{-}$ tem o maior valor representado consideramos que : \\

        $round(X) = X_{+} = \infty$

      \subsubsection*{Round towards zero}
        Como $X = (2^{130}) > 0$ então $round(X) = X_{-}$

      \subsubsection*{Round to nearest}
        Não é possível comparar X com $\infty$ portanto $round(X) = X_{-}$


  \newpage
  
  %%%%%%%%%%%%%%%%%%%%%%%%%%%%%%%%%%%%%%%%%%%%%%%%%%%%%%%%%%%%%%%%%%%%%%%%%%%%%%
  \section{Questão 3}
  \label{sec:question3}

    Temos que: \\

    $(1 \oplus x) = round(1 + x) = (1 + x)(1 + \delta) = 1$ pelo teorema enunciado em \cite{numericalIEEE}. Precisamos que $x \approx 0$ para não influenciar na soma, para isso na precisão single x deve assumir o menor valor positivo possível de um ponto flutuante, que é $x = +(1.000 \dots 00) \times 2^{-126}$. Esse valor é muito pequeno, e sua soma com 1 será arredondado para 1 também. \\

    Na precisão dupla temos a mesma situação, só que $x = +(1.000 \dots 00) \times 2^{-1022}$, valor menor ainda e que será somado com 1 e resultará em 1 após o arredondamento.

  %%%%%%%%%%%%%%%%%%%%%%%%%%%%%%%%%%%%%%%%%%%%%%%%%%%%%%%%%%%%%%%%%%%%%%%%%%%%%%
  \section{Questão 4}
  \label{sec:question4}

    \subsubsection*{Comutatividade}
      $(X \oplus Y) = round(X + Y)\footnote{Comutatividade aritmética} = round(Y + X) = (Y \oplus X)$, portanto a operação $\oplus$ é comutativa

    \subsubsection*{Associatividade}

      $(X \oplus (Y \oplus Z)) = (X \oplus round(Y + Z)) = round(X + (1 + Y + Z) \times (1 + \delta )) = (1 + X + (1 + Y + Z) \times (1 + \delta )) \times (1 + \delta ) = \dots$ \\

      A soma não é associativa, pois a cada arredondamento existe perda de precisão da conta original, isso faz com que se a ordem das somas variar, de uma operação para outra, o arredondamento final não seja o mesmo.


%%%%%%%%%%%%%%%%%%%%%%%%%%%%%%%%%%%%%%%%%%%%%%%%%%%%%%%%%%%%%%%%%%%%%%%%%%%%%%
